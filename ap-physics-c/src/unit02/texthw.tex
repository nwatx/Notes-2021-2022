\documentclass{scrreprt} % comment this out when done editing
% \documentclass{standalone}
% \usepackage{standalone}
\usepackage{chez}

\begin{document}

\section{Homework}

5. A train at a constant 60.0 km/h moves east for 40.0 min,
then  in  a  direction  50.0° east  of  due  north  for  20.0 min, and  then
west for 50.0 min. What are the (a) magnitude and (b) angle of its
average velocity during this trip?

\begin{align*}
	&= 40\hat{i} + 20cos(40)\hat{i} + 20sin(40)\hat{j} - 50 \hat{i} \\
	&= 5.32089\hat{i} + 12.8558\hat{j}
\end{align*}

\noindent 21. A dart is thrown horizontally with an initial speed of 
10 m/s  toward  point  P, the  bull’s-eye  on  a  dart  board. It  hits  at
point Q on the rim, vertically below P, 0.19 s later. (a) What is the
distance PQ?  (b)  How  far  away  from  the  dart  board  is  the  dart
released? \newline

a) Solving for $\hat{i}$ component we get $10\cdot 0.19 = 1.9 m$, then solving
for $\hat{j}$ we get

$$\Delta x = \frac{1}{2}at^2 = \frac{1}{2}(-9.8)(0.19)^2=-0.17689$$

So the distance $\vec{PQ} = |-0.17689| = 0.17689$

b) Then, the distance $\vec{PQ}$ must be $\sqrt{0.17689^2 + 1.9^2} = 1.90822 m$\newline

\noindent 23. A projectile is fired horizontally from a gun that is 
45.0 m above flat ground, emerging from the gun with a speed of
250 m/s. (a) How long does the projectile remain in the air? (b) At
what  horizontal  distance  from  the  firing  point  does  it  strike  the
ground? (c) What is the magnitude of the vertical component of its
velocity as it strikes the ground? \newline

a) $$\Delta x = -45 = \frac{1}{2}at^2 = \frac{1}{2}(-9.8)t^2$$
$$t = \pm 3.0305$$, so the projectile must remain $3.0305$ seconds in the air.

b) $3.0305\cdot 250 = 757.625 m$

c) $v = v_0 + at$ and $v_0 = 0, a = 9.8, t = 3.0305$, so $$v = -9.8\cdot 3.0305 = -29.6989 m/s$$

\noindent25. The current world-record motorcycle jump is 77.0 m,
set  by  Jason  Renie. Assume  that he  left  the  take-off  ramp  at  
12.0º to the horizontal and that the take-off and landing 
heights are the same. Neglecting air drag, determine his take-off
speed.

$$77=\frac{v_0^2\sin(2\theta)}{g}$$

$$v = 43.0727$$

\noindent27. A certain airplane has a
speed  of  290.0 km/h  and  is  diving
at  an  angle  of 30.0° below  the
horizontal  when  the pilot  releases
a radar decoy (Fig. 4-33). The hori-
zontal distance between the re-
lease  point  and  the  point  where
the decoy strikes the ground is $d =
700 m$. (a) How long is the decoy in
the  air?  (b)  How  high  was  the  re-
lease point?

\end{document}