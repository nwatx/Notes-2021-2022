\documentclass{scrreprt} % comment this out when done editing
% \documentclass{standalone}
% \usepackage{standalone}
\usepackage{chez}

\begin{document}

\section{Homework}

5. A train at a constant 60.0 km/h moves east for 40.0 min,
then  in  a  direction  50.0° east  of  due  north  for  20.0 min, and  then
west for 50.0 min. What are the (a) magnitude and (b) angle of its
average velocity during this trip?

\begin{align*}
	&= 40\hat{i} + 20cos(40)\hat{i} + 20sin(40)\hat{j} - 50 \hat{i} \\
	&= 5.32089\hat{i} + 12.8558\hat{j}
\end{align*}

21. A dart is thrown horizontally with an initial speed of 
10 m/s  toward  point  P, the  bull’s-eye  on  a  dart  board. It  hits  at
point Q on the rim, vertically below P, 0.19 s later. (a) What is the
distance PQ?  (b)  How  far  away  from  the  dart  board  is  the  dart
released?

a) Solving for $\hat{i}$ component we get $10\cdot 0.19 = 1.9 m$, then solving
for $\hat{j}$ we get

$$\Delta x = \frac{1}{2}at^2 = \frac{1}{2}(-9.8)(0.19)^2=-0.17689$$


\end{document}