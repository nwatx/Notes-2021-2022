\documentclass{scrreprt} % comment this out when done editing
% \documentclass{standalone}
% \usepackage{standalone}
\usepackage{chez}

\begin{document}

\section{Unit 03 - Forces}

\subsection{September 20, 2021}

\begin{definition}[Force]
	Any interaction between any two objects.
\end{definition}

\begin{definition}[Fundamental Forces]
	Fundamental forces:

	\begin{itemize}
		\item Gravitational force - by far the weakest force.
		\item Electromagnetic force - the fundamental force that dictates ~98\%
		of forces you experience on an everyday basis. Holds together molecules.
		Ex. rub a balloon on your head.
		\begin{itemize}
			\item People are mostly made up of empty space.
			\item Why do you feel the desk? Photon-photon interactions.
		\end{itemize}
		\item Strong Nuclear Force - since atoms are all charged positively
		in the nucleus and should theoretically repel each other, they still stick
		together because of strong nuclear force.
		\begin{itemize}
			\item Strongest force. Only works over small distances.
			\item Making these using $E=mc^2$ then it can
			easily turn into a nuclear reaction.
			\item To make a nuclear fusion reaction, two atoms must collide with large amounts
			of energy. 
		\end{itemize}
		\item Weak Nuclear Force - underlies radioactivity and decay
		\begin{itemize}
			\item The effective range of the weak force is limited to subatomic
			distances, and is less than the diameter of a proton.
		\end{itemize}
	\end{itemize}
\end{definition}

Newton's Three Forces

\begin{itemize}
	\item An object at rest stays at rest and an object in motion stays in
	motion with the same speed and in the same direction unless acted upon by an
	unbalanced force. Aka inertia - laziness.
	\item $\sum F = ma$
	\item Newton’s third law states that when two bodies interact, they apply
	forces to one another that are equal in magnitude and opposite in direction.
\end{itemize}

\end{document}