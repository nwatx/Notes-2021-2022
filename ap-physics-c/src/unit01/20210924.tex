% \documentclass{scrreprt} % comment this out when done editing
\documentclass{standalone}
% \usepackage{standalone}
\usepackage{chez}
\usepackage{pgfplots}
% We will externalize the figures
\usepackage{tikz}

\NewDocumentCommand{\evalat}{sO{\big}mm}{%
  \IfBooleanTF{#1}
   {\mleft. #3 \mright|_{#4}}
   {#3#2|_{#4}}%
}

\begin{document}

\subsection{August 24, 2021 - Derivatives}

\begin{example}[Parabolic]

Assume the graph is something generally parabolic, such as $f(t) = t^2$.

First, recall the second kinematic

$$
\Delta x = v_0t + \frac{1}{2}at^2
$$

Recall that for our tangent graphs (without the use of much calculs), we would
find the slope of a secant line. We would define the slope as

$$
v = \frac{\Delta x}{\Delta t}
$$

which would differ from the instantaneous velocity for the vast majority of the
time.

As a result, we take

$$
v = \lim_{\Delta x \rightarrow 0}\frac{x_f - x_i}{t_f - t_i}
$$

for

$$
\lim_{\Delta x \rightarrow 0}\frac{5(t + \Delta t)^2 - 5t^2}{t + \Delta t - t}
= \lim_{\Delta x \rightarrow 0}5\Delta t + 10t = 10t
$$

notice how this equation looks similar to $v = v_0 + at$ 

\end{example}

\begin{remark}[Historical Context - Leibniz and Newton]
	Derivative - coined by "Leibniz."
	
	Issues with naming it this way: derivative implies derivation, although the
	process itself of taking a derivative is differentation.

	$\Delta x$ was renamed from $\Delta x$ to $dx$. The only real velocity equation
	according to most univeristy professors is $$v = \frac{dy}{dx} = \frac{dx}{dt}$$

	Then, the acceleration equation becomes
	$$a = \frac{dv}{dt} = \frac{d}{dt}v = \frac{d^2y}{dx^2}$$
\end{remark}

\begin{remark}[Rules of Differentiation]
	A derivative is an expression that represents the rate of change of a function
	with respect to an independent variable.

	\begin{enumerate}
		\item Constant Rule. Example: if $x=6$, then $\frac{dy}{dx}=0$.
		\item Power rule: $$\frac{d}{dx}Cdt^n = (C \cdot n)t^{n-1}$$
	\end{enumerate}
\end{remark}

\begin{example}[Example]
	What is the squirrel's acceleration at $t=1$ second if its position is
	given by the equation $x = 2t^5 - 3t^2 + 2t - 4$?

	$$\frac{d}{dx} = 10t^4 - 6t + 2 = 40t^3 - 6$$
	$$\evalat{40t^3 - 6}{t = 1} = 34 m / s$$
\end{example}

\subsection{August 25, 2021}

\subsubsection{Setting up the TI-89}

\begin{remark}[TI-89 Setup]
\begin{enumerate}
	\item Hit option, and notice that there is an apps desktop.
	\item Toggle over, and select off. Then, it will take you
	directly to the screen to do calculations.
	\item If you type in $\frac{3}{7}$, it will return the same thing. There
	are two modes: approximate and exact. You will change this to approximate.
	\item Now, if you type in $\frac{3}{7}$ you should get $0.42857142857142855$.
	\item Toggle display digits from float 6 to float 10.
	\item Make sure your calculator is in degree mode.
	\item Turn on pretty print.
\end{enumerate}
\end{remark}

\subsubsection{Calculating with the TI-89}

\begin{remark}[TI-89 calculations]
\begin{enumerate}
	\item We type in our function after putting it into F3 mode (calculus).
	\item If we type in `3t' the TI-89 treats that as one variable. Instead,
	type $3 \cdot t$.
	\item $x = 3t^3$.
	\item Since we want the derivative with respect to time, append a comma after
	your function and then type what to take it with respect to.
	\item Our calculator should then output $$\frac{d}{dx} 3t^3 = 9t^2$$
	\item To clear your screen, do ``F1 + 8''
	\item TI-89s store variables across calculations. It's a good idea to clear
	your calculation before you start computing things.
\end{enumerate}
\end{remark}

\begin{example}[Sketching graphs]
	$$y = 3t^5-6t^2 = 3t^2(t^3 - 2)$$

	\begin{enumerate}
		\item Find intercepts of the function.
		\begin{enumerate}
			\item Hit F2 to get the calculator to do algebra. Hit SOLVE.
			\item Directly type the equation above into the the calculator.
			\item Hit SOLVE, and include your implicit operator $t$.
			\item Ask it to solve for $t$
			\item The resulting roots are $0, \sqrt{2}, -\sqrt{2}$.
			\item These should be easy to graph now.
		\end{enumerate}
		\item Find critical points:
		\begin{enumerate}
			\item $$\frac{dy}{dt} = 15t^4 - 13t^3 = 0$$
			\item Our resulting points are: $(-1.1, 3.15), (0,0), (-1.1, -3.15)$
		\end{enumerate}
		\item Find whether they are maximum or minimum or neither.
		\begin{itemize}
			\item We can use the first derivative or second derivative test.
		\end{itemize}
		\item Find points of inflection
		\begin{itemize}
			\item $(0.774, -1.9)$
			\item $(-0.774, 1.9)$
		\end{itemize}
	\end{enumerate}

	\begin{center}
	\begin{tikzpicture}
		\begin{axis}
		\addplot[color=blue, domain=-0.5:1.4, range=-5:5, samples=100]{3*x^5-6*x^2};
		\end{axis}
	\end{tikzpicture}
	\end{center}
\end{example}

\subsection{August 26, 2021 - Integration}

\begin{example}
	$$x = t^3 - 3t + 2$$

	\begin{enumerate}
		\item Find intercepts, $0 = t^3 - 3t + 2$
		\item Find critical points. $\frac{dx}{dt}=3t^2-3=0$. This returns
		$(-1, 4), (1,0)$
		\item Check Concavity

		$$\frac{d^2x}{dt^2} = 6t$$

		\item And set the resulting points to the critical point values. Evaluate,
		and determine their resulting sign.
		\item Find points of inflection.
		
		$$0 = \frac{d^2x}{dt^2} = 6t, t = 0$$


	\end{enumerate}

	\begin{center}
	\begin{tikzpicture}
		\begin{axis}
		\addplot[color=blue, domain=-3:3, range=-5:5, samples=100]{x^3 - 3 * x + 2};
		\end{axis}
	\end{tikzpicture}
	\end{center}
\end{example}

\begin{example}[Integration 1]
	Let $a = 6t$, then $$\int \frac{d}{dt} 6t = 3t^2 + C$$

	What does this actually do? Solve for the area under the curve.
\end{example}

\begin{example}[Integration 2]
	If $v = 10t^2$, what is the displacment of the object at time $t = 5$?

	$$
	\int \frac{d}{dt}_0^5 10t^2 =
	\evalat{\frac{10}{3}t^3}{t=5} - \evalat{\frac{10}{3}t^3}{t=0}
	\approx 416.667
	$$
\end{example}

Some things you may have learned from Calculus AB/BC

$$\sum dx = \sum {v \cdot dt}$$
$$x = \int v \cdot dt$$

\subsection{August 27 - Kinematics}

\begin{definition}[Kinematics]
	\begin{itemize}
		\item 1st Kinematic = $$\vec{v} = \vec{v}_0 + \vec{a} \cdot t$$
		\item 2nd Kinematic = $$\Delta x = v_0t + \frac{1}{2}at^2$$
		\item 3rd Kinematic = $$\vec{v}^2 = \vec{v_0}^2 + 2\vec{a}\Delta \vec{x}$$
		\item 4th Kinematic = $$\Delta \vec{x}= \frac{1}{2}(v + v_0)\cdot t = \int_{t=a}^{t=b}f(x)dx$$
	\end{itemize}
\end{definition}

\end{document}