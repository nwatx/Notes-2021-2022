% \documentclass{standalone}
\documentclass{scrreprt} % comment this out when done editing
\usepackage{chez}

\begin{document}

\section{Unit 1}

\subsection{August 18, 2021}

\subsubsection{Mr. Clifford's Introduction Video}

\begin{itemize}
	\item Scarcity is the idea that there are limited resources.
	\item Example: Thursday night flight for \$275, \$300, \$325 dollar
	flight. The most expensive flight was actually the cheapest because of
	opportunity cost: she could work at the restaurunt for the next few days.
	\item Why don't we produce all the phones in our country?
	\item Unseen costs: because of costs in the United States, phones are
	much cheaper because China manufacturing is significantly cheaper in the
	United States. Economists often oppose tarriffs for this reason.
\end{itemize}

\begin{definition}[Economics 1]
	The study of how people interact with each other and with their natural
	surroundings in producing their livelihoods, and how this changes over time.
\end{definition}

\begin{definition}[Economics 2]
	Economics is about how individuals, businesses, governments, and nations
	make choices about how to allocate limited resources when faced with
	unlimited wants and needs.
\end{definition}

In essence, economics solves the problme of scarcity - which occurs when there
are limited quantities of things.

\begin{definition}[Resources]
	Anything that can be used to produce something else.
\end{definition}

\subsubsection{The Factors of Production}

\begin{itemize}
	\item Labor: the time and effort that people devote to producing goods
	and services.
	\begin{itemize}
		\item Human capital: the knowledge and skill base of workers in society
		from education, trainings, and experience.
		Investing in human capital can increase your productive capacities as
		an economy. McDonald's and many companies will pay for associated colleges.
	\end{itemize}
	\item Capital: Items used to produce goods and servies: machines, tools,
	buildings. NOT: money, stocks.
	\item Land: Nature that we use to produce goods and services: water, animals, minerals
	\item Entrepreneurship: resource that decides how to organize the land, labor.
	and capital of production; makes decisions and bears risks.
\end{itemize}

\begin{remark}
	Technology falls under capital, since it is able to produce goods and services.
	In the future, this may fall under its own category because they are able
	to produce larger things.
\end{remark}

\subsubsection{How does scarcity impact human behavior?}

People must make choices about how to use scarce resources.

\begin{definition}[Tradeoffs]
	All alternatives that are considered when making a choice.
\end{definition}

\begin{definition}[Opportunity Cost]
	The value of the next best alternative when making a decision.
\end{definition}

Example: by watching a movie at home instead of going out with friends, you are trading
the benefits of added stress of going outdoors.

How do individuals make the best decisions?

\begin{itemize}
	\item Benefits: the added satisfaction and benefit when a good is purchased.
	\item Costs: The added costs (and opportunity costs) when a good is purchased.
\end{itemize}

\subsubsection{Should a city host the olympics?}

\begin{itemize}
	\item How are the land, labor, capital, and entrepeneurship used if
	you host the olympics.
	\begin{itemize}
		\item Venues cost \$ 7.6 billion. Bids are usually an optimistic estimate.
		\item One year postpoponent meant that there were many costs added.
		\item New costs are added incurred around 2.5 billion dollars to the games' budget.
		\item COVID also is taking away ability to recoup that money.
		; There will be no income from visitors because of COVID.
		\item Around 2 billion dollars lost in economic benefits.
		\item Corporate sponsors contribute billions of dollars and the IOC
		committed around 600 million. The overall cost of the pandemic is much
		higher than the olympics.
		\item Japan has spent over 800 billion dollars on stimulus packages.
	\end{itemize}
	\item What's something interesting you heard?
	\begin{itemize}
		\item First game to be postponed.
		\item Most expensive summer game ever.
		\item Cost over 20 billion by the end.
	\end{itemize}
	\item What are the trade-offs of hosting?
	\begin{itemize}
		\item You often lose money.
		\item There would be no spectators this year because of COVID-19.
	\end{itemize}
\end{itemize}

\subsection{August 23, 2021}

Discuss what your terms mean, adn then come up with a group motion to represent
this term.

\begin{enumerate}
	\item Economics
	\item Scarcity
\end{enumerate}

\end{document}