\documentclass{scrreprt} % comment this out when done editing
% \documentclass{standalone}
% \usepackage{standalone}
\usepackage{chez}

\begin{document}

\section{Unit 02}

Knowing the different types of unemployment and when they occur is an integral
part of the AP Macroeconomics syllabus. Thus, understanding the characteristics
of the different types of unemployment is necessary. Unemployment comes in three
forms: cyclical, structural, and frictional.

\begin{enumerate}
	\item  Cyclical unemployment is the result of a downturn in the business cycle.
This type of unemployment is temporary and only occurs when there is a contraction in the economy.
An example of cyclical unemployment is someone who is laid off because a company isn't selling enough due to a recession and can't afford to pay them.

	\item Structural Unemployment - Structural unemployment occurs when the skills of workers do not meet the skills demanded of such workers by employers.
This type of unemployment often occurs as a result of technological change, resulting in workers’ skills becoming obsolete.
An example of structural unemployment is a typewriter repairman. While they may have been necessary in the 1950s, nobody has a typewriter anymore and thus the typewriter repairman is out of a job.
	\item Frictional Unemployment
\begin{itemize}
	\item Workers are between jobs
	\item Workers have just entered the workforce and are yet to find a job.
	\item An example of someone who is frictionally unemployed is someone who quits a job and is looking for a new one.
	\item Natural Rate of Unemployment
	\item Finally, the concept of the natural rate of unemployment is the key to
	understanding unemployment in the macroeconomy
\end{itemize}
\end{enumerate}

\begin{remark}

The natural rate of unemployment is defined as the combination of structural and frictional unemployment present in an efficient expanding economy in macro equilibrium.
Basically, it means that there is no cyclical unemployment, not no unemployment at all. There will ALWAYS be some frictional and structural unemployment, as there will always be people in between jobs or without necessary skills for a job
Why is this? Well, it makes sense, you can never employ EVERY member of a labor force because some people might quit their job and be frictionally unemployed or someone might not be able to get a job with their skills. In fact, a 0\% unemployment rate would be a red flag to any economist.
In the United States, this rate is between 4-6\%, which is why aiming for a 0\% rate of unemployment is dangerous to the health of our economy.
For the AP Exam, typically, unemployment from 0\% - 3 or 4\% implies an inflationary gap, 4-6\% is full employment, 7\%+ is a recessionary gap (they will make it glaringly obvious, something like either 2\% or 10\%, or they might flat out say "recession")
At 0\% unemployment, inflation would be high, rising prices

\end{remark}

\end{document}