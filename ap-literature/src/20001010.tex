\documentclass{scrreprt} % comment this out when done editing
\usepackage{chez}

\begin{document}

\section{Poetry}

\subsection{August 18, 2021}

Poetry tips

\begin{definition}[Speaker]
	The "narrator" of a poem.
\end{definition}

\begin{itemize}
	\item Read first for the literal.
	\item To figure out the literal pay attention to punctuation.
	\item The tone/attitude
	towards the subject is key.
\end{itemize}
\begin{example}[Clint Smith - Something you should know]

Something You Should Know

\begin{quote}
	is that as a kid, I once worked at a pet store.
	
	I cleaned the cages
	
	of small animals like turtles, hamsters,
	
	rabbits, and hermit crabs. 
	
	I watched the hermit crab continue
	
	to grow, molt, shed its skin and scurry across
	
	the bottom of the aquarium to find a new shell.
	
	Which left me afraid for the small creature,
	
	to run around all exposed that way, to have
	
	to live its entire life requiring something else
	
	to feel safe. Perhaps that is when I became afraid
	
	of needing anything beyond myself. Perhaps
	
	that is why, even now, I can want so desperately
	
	to show you all of my skin, but am more afraid
	
	of meeting you, exposed, in open water.
\end{quote}

Observations

\begin{itemize}
	\item Clint Smith analogizes himself to the hermit crab.
	\item He is scared of showing the person his skin, but it would require him
	to do something presumably uncomfortable.
	\item He is scared to go outside of his own boundaries.
	\item He wants to share his story, but is afraid to do so because he feels
	uncomfortable doing so.
\end{itemize}
\end{example}

\begin{itemize}
	\item Identify and describe what specific textural details reveal about a
	character, that character's perspective, and their motives.
	\item Identify and describe the narrator or speaker of a text.
	\item Identify and explain the function of point of view in a narrative.
\end{itemize}

\subsection{August 23, 2021}

\begin{example}[Sticks - George Saunders]

Every year Thanksgiving night we flocked out behind Dad as he dragged the Santa suit to the road and draped it over a kind of crucifix he'd built out of metal pole in the yard. Super Bowl week the pole was dressed in a jersey and Rod's helmet and Rod had to clear it with Dad if he wanted to take the helmet off. On the Fourth of July the pole was Uncle Sam, on Veteran’s Day a soldier,  on Halloween a ghost. The pole was Dad's only concession to glee. We were allowed a single Crayola from the box at a time. One Christmas Eve he shrieked at Kimmie for wasting an apple slice. He hovered over us as we poured ketchup saying: good enough good enough good enough. Birthday parties consisted of cupcakes, no ice cream. The first time I brought a date over she said: what's with your dad and that pole? and I sat there blinking.

We left home, married,  had children of our own, found the seeds of meanness blooming also within us. Dad began dressing the pole with more complexity and less discernible logic. He draped some kind of fur over it on Groundhog Day and lugged out a floodlight to ensure a shadow. When an earthquake struck Chile he lay the pole on its side and spray painted a rift in the earth. Mom died and he dressed the pole as Death and hung from the crossbar photos of Mom as a baby. We'd stop by and find odd talismans from his youth arranged around the base: army medals, theater tickets, old sweatshirts, tubes of Mom's makeup. One autumn he painted the pole bright yellow. He covered it with cotton swabs that winter for warmth and provided offspring by hammering in six crossed sticks around the yard. He ran lengths of string between the pole and the sticks, and taped to the string letters of apology, admissions of error, pleas for understanding, all written in a frantic hand on index cards. He painted a sign saying LOVE and hung it from the pole and another that said FORGIVE? and then he died in the hall with the radio on and we sold the house to a young couple who yanked out the pole and the sticks and left them by the road on garbage day.

\end{example}

Contributor's note

In the contributor's notes in "Story" magazine, George Saunders writes, "For two years I'd been driving past a house like the one in the story, imagining the owner as a man more joyful and self-possessed and less self-conscious than myself. Then one day I got sick of him and invented his opposite, and there was the story."


\end{document}