\documentclass{scrreprt} % comment this out when done editing
\usepackage{chez}

\begin{document}

\section{Poetry}

\subsection{August 18, 2021}

Poetry tips

\begin{definition}[Speaker]
	The "narrator" of a poem.
\end{definition}

\begin{itemize}
	\item Read first for the literal.
	\item To figure out the literal pay attention to punctuation.
	\item The tone/attitude
	towards the subject is key.
\end{itemize}
\begin{example}[Clint Smith - Something you should know]

Something You Should Know

\begin{quote}
	is that as a kid, I once worked at a pet store.
	
	I cleaned the cages
	
	of small animals like turtles, hamsters,
	
	rabbits, and hermit crabs. 
	
	I watched the hermit crab continue
	
	to grow, molt, shed its skin and scurry across
	
	the bottom of the aquarium to find a new shell.
	
	Which left me afraid for the small creature,
	
	to run around all exposed that way, to have
	
	to live its entire life requiring something else
	
	to feel safe. Perhaps that is when I became afraid
	
	of needing anything beyond myself. Perhaps
	
	that is why, even now, I can want so desperately
	
	to show you all of my skin, but am more afraid
	
	of meeting you, exposed, in open water.
\end{quote}

Observations

\begin{itemize}
	\item Clint Smith analogizes himself to the hermit crab.
	\item He is scared of showing the person his skin, but it would require him
	to do something presumably uncomfortable.
	\item He is scared to go outside of his own boundaries.
	\item He wants to share his story, but is afraid to do so because he feels
	uncomfortable doing so.
\end{itemize}
\end{example}

\end{document}